\documentclass[]{article}
\usepackage{lmodern}
\usepackage{amssymb,amsmath}
\usepackage{ifxetex,ifluatex}
\usepackage{fixltx2e} % provides \textsubscript
\ifnum 0\ifxetex 1\fi\ifluatex 1\fi=0 % if pdftex
  \usepackage[T1]{fontenc}
  \usepackage[utf8]{inputenc}
\else % if luatex or xelatex
  \ifxetex
    \usepackage{mathspec}
    \usepackage{xltxtra,xunicode}
  \else
    \usepackage{fontspec}
  \fi
  \defaultfontfeatures{Mapping=tex-text,Scale=MatchLowercase}
  \newcommand{\euro}{€}
\fi
% use upquote if available, for straight quotes in verbatim environments
\IfFileExists{upquote.sty}{\usepackage{upquote}}{}
% use microtype if available
\IfFileExists{microtype.sty}{%
\usepackage{microtype}
\UseMicrotypeSet[protrusion]{basicmath} % disable protrusion for tt fonts
}{}
\ifxetex
  \usepackage[setpagesize=false, % page size defined by xetex
              unicode=false, % unicode breaks when used with xetex
              xetex]{hyperref}
\else
  \usepackage[unicode=true]{hyperref}
\fi
\hypersetup{breaklinks=true,
            bookmarks=true,
            pdfauthor={},
            pdftitle={},
            colorlinks=true,
            citecolor=blue,
            urlcolor=blue,
            linkcolor=magenta,
            pdfborder={0 0 0}}
\urlstyle{same}  % don't use monospace font for urls
\setlength{\parindent}{0pt}
\setlength{\parskip}{6pt plus 2pt minus 1pt}
\setlength{\emergencystretch}{3em}  % prevent overfull lines
\setcounter{secnumdepth}{0}

\date{}

\begin{document}

\section{Reader Response for Peer
Review}\label{reader-response-for-peer-review}

\textbf{A poisonous mix of inequality and sluggish wages threatens
globalisation}

Globalisation is under attack by protectionist governments in response
to disgruntled voters who are disillusioned by the growing income
disparity they are experiencing. The author asserts that businesspeople
and policymakers who enjoy the prosperity brought about by globalisation
should act to defend it despite the impact on those who are affected
adversely. They believe that the economic benefits outweight the social
drawbacks.

I disagree with their stance that a mobile society is better than an
equal one. Class conflict will exist as long as the large wealth gaps
between the rich and the poor exist. It is wrong that a person should
live in poverty regardless of their willingness or ability to work while
the wealthy should live wasteful and extravagant lives. The authors
believe that this should be tolerated as it is good for the economy as a
whole.

According to Easterling (2003), `capitalism is an economic system that
is inherently crisis-prone'. It is a system that necessitates conflict.
Particularly, conflict between the `haves' and the `have-nots' which is
further exacerbated by the fact that the wealthy wield control over the
resources the poor require for survival trapping them in a cycle of wage
slavery. Thus, I argue that the societal mobility the authors support
does not imply mobility for all but only the select aristocratic elite
and is hence destructive.

However, I do agree with the authors' suggestion that the implementation
of compassionate and egalitarian policies that cushion the impact of
globalisation on the working class is a step towards a progress even if
the goals of the author are motivated by the desire to protect
globalisation. These progressive policies such as re-training programs,
the decoupling of pension schemes, and the persistence of medical
benefits after retrenchment will definitely help to equalise society.
Nevertheless, it is dubious that corporations would adopt such a stance
as the drain on their profits would be substantial.

The sentiment is not only shared by the authors of the article but also
by Claudia Juech, Associate Vice President of the Rockerfeller
Foundation, a philanthropic organisation working to `improve the
well-being of humanity around the world'. In an article published in the
World Economic Forum, she posits that these shifts in the labour market
is not entirely a crisis, but an opportunity to revise the paradigm of
employment in modern society (Juech, 2015). She asserts that the
detachment of benefits can provide safety nets for the economically
vulnerable and improve the efficiency of the workforce.

In conclusion, social and economic equality should not be compromised in
favour of societal mobility as large market movements that widen the gap
between the rich and the poor is inherently destructive. Instead, upper
management of corporations and policymakers should be acting with the
the interests of society as a whole by implementing progressive programs
and policies to ensure that globalisation improves the standard of
living for everyone in all tiers of society.

\subsection{Sources}\label{sources}

\begin{enumerate}
\def\labelenumi{\arabic{enumi}.}
\itemsep1pt\parskip0pt\parsep0pt
\item
  Easterling, S. (2003, November/December). International Socialist
  Review. Retrieved February 11, 2016, from
  http://isreview.org/issues/32/crisis\_theory.shtml
\item
  Juech, C. (2015, January 18). The opportunities of the changing
  workforce. Retrieved February 11, 2016, from
  http://www.weforum.org/agenda/2015/01/the-opportunities-of-the-changing-workforce/
\item
  The Economist,. (2007). Rich man, poor man. Retrieved 11 February
  2016, from http://www.economist.com/node/8554819
\end{enumerate}

\end{document}
